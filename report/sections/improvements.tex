\section{Future improvements}
% Tackle the most obvious first
The result of this project can be improved in several ways: the most
obvious way is spending the necessary time to finalize all software
nodes and their interactions such that all milestones defined in
section~\ref{sec:milestones} are reached.

% Spend some time to ditch the additional hardware
The current robot system currently depends on some external hardware
which would be unnecessary if the Linux kernel was aware of all the
peripherals offered by the hardware. It may be worth it to investigate
why these are not already supported by NixOS and ultimately push he
eventual fixes upsteam. While improving the overall quality of this
project, it would also bump the overall quality of NixOS itself.

% An embedded system instead
An alternative to the current hardware setup and software stack would be
the design of a truly embedded system without a rich OS (Linux in this
case). A custom hardware design would allow the installation of multiple
sensors that could help in the problem space: such as a compass for
deriving the direction in which the robot is currently heading; and
directly designing the wideband-module of the DWM into the hardware, so
that the whole DWM-tag subsystem can be replaced. Further, an embedded
system would allow the implementation of a code-base that fully relies
on hardware interrupts\footnote{A feature more prominent and earier to
  work with on an embedded system, where a change in input voltage
  directly interrupts the processor (be it from a task of lower
  priority, or from sleep) to start a procedure to process data.} which
would enable the removal of any busy-while loops related to I/O. the
system would thus be made real-time capable if properly programmed. An
interrupt-oriented approach would allow the processor to sleep whenever
it has no data to process. This allows for a more responsive system, and
one that consumes less energy to operate.

% While not required, it would be more proper
Although the problem description does not call for a real-time system,
such an implementation would be much more proper for the problem domain,
and would more closely approximate how an actual industial system would
be implemented.

% Possible to apply formal analysis
Further, the implementation of such a system could be the subject for a
formal analysis with the lack of a rich OS that needlessly complicates
the call-stack.

% Can we port the current software stack?
The software-node based structure can readily be ported to a real-time
operating system (RTOS) if it allows the implementation of ``tasks''
that trigger on interrupts; and/or can be called from another task; and
if the RTOS implements message passing between these tasks. An example
for such a RTOS would be Real-Time Interrupt-driven Concurrency (RTIC)
\parencite{rtic}. Of possible interest would also be the MicroPython
project, which could allow minimal porting changes of the currently
available Python nodes if properly implemented for the target RTOS.
\parencite{micropython}.
