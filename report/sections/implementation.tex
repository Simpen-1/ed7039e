\section{Implementation}
\subsection{Milestones}
The project is divided into four milestones:
\begin{enumerate}
\item \textbf{Two-dimensional navigation:}
  the system should be able to determine its coordinates in a ad-hoc, local grid.
  From its initial position, it should then be able to respond to movement commands on the form ``move to position $(x, y)$''.

\item \textbf{Navigation-line detection:}
  using the subsystem for two-dimensional navigation, the system is to cross a line on the floor,
  thus detecting it and follow it towards the station.

\item \textbf{Station proximity detection, object pickup:}
  once the navigation-line is being followed, the system is to sense when it is sufficiently close to the station to readily use its arm to pick the object up.

\item \textbf{Object displacement, dropp-off:}
  after the object has been picked up, the system is to move to another station, find its navigation-line, follow it, and drop the object.
  Note that this milestone is a permutation of the combination of the previous milestones: the same phases should be done in the same order,
  but the system is to move to the second station instead and execute the pickup-process in reverse.
\end{enumerate}

\subsection{Prototyping}
% Here we describe the prototyping stages of the system's components if anything
% out of the ordinary pops up.
\subsubsection{Decawave}
% (X, Y, Z, Q); how is Q calculated?
% What should we do if we cannot connect to 4 anchors at once, a wait?
% Mention that:
% - we have to account for the fact when we tag cannot connect to at least 3 anchors.
% - Qualitative data depends a lot on the positioning of the anchors
% - Built-in 3-axis accelerometer
% - Raspberry Pi compatible GPIO header. Communication via UART.
% - How should we interpset data? It is random proccess? Can we consider noise gaussian?

\subsubsection{BrickPi3}
% Talk about the confusing installation scipt(s) that most likely installs a
% lot of unecessary components.
% Mention that they were boiled down into `brickpi.nix`.
% Note: only the brickpi3 library is required. Remainder unused.

\subsubsection{NixOS}
% Problems with SPI on the Raspberry Pi.
% Link to <https://github.com/NixOS/nixpkgs/pull/79370> for credit.
% Compare approach with Raspbian: mention that only a raspbian fork is officially supported for the BrickPi3.