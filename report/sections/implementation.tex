\section{Implementation}
The git repository used throughout this project is publicly available at \href{https://github.com/tmplt/ed7039e}{Github/tmplt/ed7039e} \parencite{repo}.
If not otherwise specified, any references to a repository shall mean this repository.

\subsection{Milestones}
The project is divided into four milestones:
\begin{enumerate}
\item \textbf{Two-dimensional navigation:}
  the system should be able to determine its coordinates in a ad-hoc, local grid.
  From its initial position, it should then be able to respond to movement commands on the form ``move to position $(x, y)$''.

\item \textbf{Navigation-line detection:}
  using the subsystem for two-dimensional navigation, the system is to cross a line on the floor,
  thus detecting it and follow it towards the station.

\item \textbf{Station proximity detection, object pickup:}
  once the navigation-line is being followed, the system is to sense when it is sufficiently close to the station to readily use its arm to pick the object up.

\item \textbf{Object displacement, dropp-off:}
  after the object has been picked up, the system is to move to another station, find its navigation-line, follow it, and drop the object.
  Note that this milestone is a permutation of the combination of the previous milestones: the same phases should be done in the same order,
  but the system is to move to the second station instead and execute the pickup-process in reverse.
\end{enumerate}

\subsection{Prototyping}
% Here we describe the prototyping stages of the system's components if anything
% out of the ordinary pops up.
\subsubsection{Decawave}
% (X, Y, Z, Q); how is Q calculated?
% What should we do if we cannot connect to 4 anchors at once, a wait?
% Mention that:
% - we have to account for the fact when we tag cannot connect to at least 3 anchors.
% - Qualitative data depends a lot on the positioning of the anchors
% - Built-in 3-axis accelerometer
% - Raspberry Pi compatible GPIO header. Communication via UART.
% - How should we interpset data? It is random proccess? Can we consider noise gaussian?

\subsubsection{BrickPi3}
% Talk about the confusing installation scipt(s) that most likely installs a
% lot of unecessary components.
% Mention that they were boiled down into `brickpi.nix`.
% Note: only the brickpi3 library is required. Remainder unused.

% What need to be done to be able to stack BrickPi3s?

\subsubsection{NixOS}
% Summarize what NixOS is and what it aims to provide
NixOS is a Linux-distribution with the ability to declare a certain system in a functional manner.
It is buit upon the Nix package manager that, in summary, aims to be
\begin{inline-enum}
\item reproducible:
  ``packages [are built] in isolation from each other. [\ldots] they are reproducible and don't have any undeclared dependencies.''\footnote{A positive side-effect of this feature is the complete mitigation of ``dependency hell''.};
  so if a package works on one machine, it will work on any machine.
  Also, when building a package on multiple machine, all machines will yield the exact same ouput file tree.
\item declarative:
  packages are described in expressions that are trivially shared and combined with other package declarations.
\item reliable:
  ``installing or upgrading one package cannot break other packages''.
\end{inline-enum}~\parencite{nixos.org}

% How does NixOS enable the above funtionality?
To enable the preceeding features, everything provided to and by Nix is stored on a Nix-enabled system under \texttt{/nix/store/<hash>-<pkg>},
where \texttt{<hash>} is a SHA256 checksum that provides a unique identifier to the package with name \texttt{<pkg>}.
If any dependency for the package (or if the build procedure) is in any way changed, a wholly new checksum is generated.
This Nix store is mounted as read-only to make its content immutable.
A usage of any package in the Nix store is manifested in the system's file system as a symbolic link.

% How does NixOS differ from conventional distros?
Because all files that are built and ultimately used are stored under \texttt{/nix/store}, NixOS breaks the filesystem hierarchy standard (FHS).
% TODO: what does this entail?

% Problems with SPI on the Raspberry Pi.
% Link to <https://github.com/NixOS/nixpkgs/pull/79370> for credit.
% Compare approach with Raspbian: mention that only a raspbian fork is officially supported for the BrickPi3.
On Raspbian, a simple \texttt{echo 'dtparam=spi=on' >> /boot/config.txt} and system reboot enables SPI and thus the ability to communicate with the BrickPi3.
On NixOS however, this file (nor any equivalent) exists, because of the wholly different design philosophies with the Debian-based Raspbian.
The ``Nix-appoach'' is instead...

\subsubsection{RobotOS}
% While a prime platform to package a framework that depends on specific software version,
% it seems to complex to package for NixOS.

\subsubsection{LCM}
