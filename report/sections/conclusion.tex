\section{Conclusions}
For building a robot supposed to complete the task given in this project, the conclusion is that lego is not the best choice. Many of the lego parts are flexible and are connected to each other in a way that makes the robot verry ''wobbly''. The accuracy of the robot arm was heavily reduced due to the flexibility of some lego parts. Also from section \ref{results}, it can be seen that the fastest controller for the robot arm in terms of rise time and settling time was in fact the P controller for the vertical movement and the PD controller for horizontal movement. Despite this the included library functions from BrickPi3 for motor movement was used instead since the P and PD controllers made the robot ''wobble'' very much and there was risk of knocking the box of the industry platform. As mentioned the included library functions for motor movement was used instead since they achieved a good combination of being both quick and smooth.\\
For doing the task of taking a instruction from arrowhead, which was pick up the box or put down the box at different locations, our conclusion is that the robot could do it with some difficulties. The difficulties being that the lego flexibility reduced the accuracy of the robot arm which made the robot sometimes miss the box.
