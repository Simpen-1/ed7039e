\section{Introduction}
\subsection{Background}
% Introduce ARTEMIS and the Arrowhead project
The ARTEMIS\footnote{\textbf{A}dvanced \textbf{R}esearch \& \textbf{T}echnology for \textbf{EM}bedded \textbf{I}ntelligent \textbf{S}ystems} Industry Association is an organisation with the goal of innovating the embedded intelligent systems sector in Europe.~\parencite{artemis}
One of the approaches to reach this goal was the 2013 start of the ARROWHEAD project which ``[address] the efficiency and flexibility at the global scale by means of collaborative automation for five application verticals; [one of which being] production (manufacturing, process, energy).''~\parencite{arrowhead-project-call}

% Summarize the result of the project
The project concluded 2017 and yielded the Arrowhead Framework which enable interoperability between Internet of Things (IoT) devices by providing guarantees for:
\begin{inline-enum}
    \item real-time data handling;
    \item data and system security;
    \item automation system engineering; and
    \item scalability of automation systems.
\end{inline-enum}
\parencite{arrowhead-project-about}
With the above guarantees provided by Arrowhead, resources otherwise required for developing a new framework (ad-hoc or not) may instead be allotted to the system the framework ultimately will interact with.

% Describe the reason for this project.
% TODO: extend/rewrite; describe that the process of moving an object from a to b by use of robots have not been solved yet (mentioned during the second lecture).
The reason for this project is then to verify that Arrowhead can indeed be used to resolve a common industry problem: production; specifically the act of moving a object from one station to another,
which is a common procedure in product pipelines.

% TODO: mention that Arrowhead aims to create de-facto standards for European industrialization efforts?

% XXX: Currently disjointed from the background section: background has large focus on Arrowhead. This section does not.
\subsection{Problem description}
% Summarize the problem and describe that we're simulating a factory settings.
The problem this project aims to solve is that of automatically moving an object from a designated pick-up point to a designated drop-off point on a system-external signal.
In practise, the problem space aims to mimic the environment of a factory floor where an object under manufacture has been fully processed in a station and should be moved to the next.

% Boil the issue down to simple mathematical terms.
Described in different terms, the problem is the automatic displacement of an object from an origin point $(x_0, y_0, z_0)$ to a destination point $(x, y, z)$,
where $(x, y)$ denote a position on the factory floor and $z, z \geq 0$, an elevation above the floor.

% Mention the inclusion of Arrowhead in the problem.
A station signals a fully processed object via the Arrowhead Framework in a defined local cloud.
Upon this signal, the task of moving the object is to be delegated to the mechatronical system this project aims to develop.

% Draw a nice state machine representation of the issue.
From the above problem description five possible system states are thus argued for:
when the robot is
\begin{inline-enum}
\item waiting for a command;
\item moving to a destination;
\item following a navigation line;
\item picking an object up; and
\item dropping an object off.
\end{inline-enum}
The relation between these states, and the allowed transitions between them is visually described using the state machine in Fig.~\ref{fig:state_machine}.
\begin{figure*}[h]
  \centering
  \begin{tikzpicture}
    % TODO transpose?
    \node[state, initial, accepting] (wait) {waiting for \\ command};
    \node[state, right of=wait] (move) {moving to \\ destination};
    \node[state, below of=move] (follow) {following \\ line};
    \node[state, right of=follow] (pickup) {picking \\ up object};
    \node[state, left of=follow] (dropoff) {dropping \\ off object};

    \path[->] (wait) edge node{Arrowhead \\ signal} (move)
    (move) edge[sloped] node{line detected \\} (follow)
    (follow) edge[sloped] node{station detection \\ (system w/o object)} (pickup)
    (follow) edge[sloped] node{station detection \\ (system w/ object)} (dropoff)
    (pickup) edge[sloped] node{object picked up \\} (move)
    (dropoff) edge[sloped] node{object dropped \\} (wait)
    ;
  \end{tikzpicture}
  \caption{High-level state machine of the system.}
  \label{fig:state_machine}
\end{figure*}

% TODO: insert an image of the conveyor belt with measures.

\subsection{Limitations}
% Note and expound on the very large problem domain.
The problem domain is vast and infinite configurations can mimic a wanted factory setting:
\begin{inline-enum}
    \item different placement/rotations of one or more stations;
    \item obstacles of multiple types; static and dynamic (e.g. humans, other robots);
    \item recovery/discard of a dropped object; and
    \item conditional and/or time-dependent origin/destination points.
\end{inline-enum}
Support for more configurations require a more complex system model, simulation and implementation;
and more complexity requires more development time.

% Note time limitation and the smaller problem space we aim to tackle.
The system developed in this project is to be thoroughly modelled, simulated, and implemented in the time span of two semesters.
For that reason (and those stated in the previous paragraph), this project is thus limited to two stations:
a single designated pick-up point and a single designated drop-off point  --- both static and elevated (however, $z_0 = z$).
Furthermore, the system environment will contain no other obstacles, and if an object is dropped, the displacement procedure ends and is considered to have failed;
this project aims to implement a proof-of-concept solution for the stated problem, not a readily available product ready for industry integration.
If an object is successfully displaced automatically to its destination once, the project is considered a success.

% Note that the system has no real-time requirement
% XXX: unless we require the system to not overshoot its 2D destionations, then we need to process location data with known delay.
% TODO: find a reference on this statement.
Because the above defined problem domain lacks any time component (as would be the case if the system would need to dodge moving obstacles, respond to time-dependent destinations, etc.) and becafuse the object need only be moved to its destination eventually, the implementation is under no real-time requirement.
This greatly lessens the requirement of the system model and software implementation.

\subsection{Arrowhead framework}
The three mandatory core systems when running Arrowhead framework are; Service Registry, Orchestrator and Authorization. 
The Gateway and Gatekeeper is needed when communicating with other clouds (inter-cloud). 
All information in this section is from \url{https://github.com/eclipse-arrowhead/core-java-spring}.

\subsubsection{Service Registry}

The Service Registry holds the services and IP address and port to where the services is located. The information is stored in a SQL database. 
Service Registry have three methods, register, unregister and query. The query method translates system name into information about physical endpoint; IP address and port. 

\subsubsection{Authorization}
The Authorization system have two methods. 
Authorization control are controlling which clients are allowed to consume a service. This is controlled through intra- and inter-cloud rules. The information is stored in a SQL database. 
The other method is token generation which is used for accessing services within cloud. 

\subsubsection{Orchestrator}
The Orchestrator have two methods.
When a application requsts a service, the orchestrator will look up if the application system is allowed to access that service. 
If allowed it will send back the IP address and port. 
The other methods is used when the IP address and port are changed for a service. 
The Orchestrator will send the new information for the services which are connected. 

There are two types of orchestration; dynamic orchestration and store orchestrator. 
Dynamic orchestrator uses the application system name and authorization properites to find the providers. 
Store orchestration uses the application systems's id to search the orchestration store database for providers. 

\subsubsection{Gatekeeper}
The Gatekeeper have two methods; Global Service Discovery (GSD) and Inter-Cloud Negotiation (ICN).  
GSD is looking via Relay systems for other clouds and looks for a provider which are serving a specific service. 
After GSD the ICN process is used for establishing a way of collaboration between clouds. 

\subsubsection{Gateway}
The Gateway system have two methods; connecting to a consumer and connection to a provider. 
The two methods is used during the ICN process to establish a new datapath. 