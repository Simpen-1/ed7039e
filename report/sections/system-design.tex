\section{System design and composition}
\subsection{Model}
The system is modelled in two parts: the moving base and the\ldots
% TODO: describe the moving base, and the attachment, whatever it ends up being.
% Include physical models.

% Describe the states of the system.
We argue for the folloing five states of the system:
\begin{inline-enum}
\item waiting for a command;
\item moving to a destination;
\item following a navigation-line;
\item picking an object up; and
\item dropping an object off.
\end{inline-enum}
The relation between these five states are visually described in the state machine of Fig~\ref{fig:state_machine}.
\begin{figure*}[ht]
  \centering
  \begin{tikzpicture}
    \node[state, initial, accepting] (wait) {waiting for \\ command};
    \node[state, right of=wait] (move) {moving to \\ destination};
    \node[state, below of=move] (follow) {following \\ line};
    \node[state, right of=follow] (pickup) {picking \\ up object};
    \node[state, left of=follow] (dropoff) {dropping \\ off object};

    \path[->] (wait) edge node{Arrowhead \\ signal} (move)
    (move) edge[sloped] node{line detected \\} (follow)
    (follow) edge[sloped] node{station detection \\ (system w/o object)} (pickup)
    (follow) edge[sloped] node{station detection \\ (system w/ object)} (dropoff)
    (pickup) edge[sloped] node{object picked up \\} (move)
    (dropoff) edge[sloped] node{object dropped \\} (wait)
    ;
  \end{tikzpicture}
  \caption{High-level state machine of the system.}
  \label{fig:state_machine}
\end{figure*}

\subsection{Simulation}
% Simulte the models from the previous section and show that it will work.
% Motivate regulation approach.

\subsection{Hardware}
% Explain the raspberry pi and it's attachments.

\subsection{Software}
\subsubsection{Reproducible system image generation}
% Explain the repo's *.nix files and what they do
The system image of the Raspberry Pi is generated via the repository's \texttt{mmc-image.nix} file ---
an auxiliary \texttt{build.sh} script is available to generate and subsequently flash a target storage device in a single command execution.
\texttt{mmc-image.nix} contains an expression of the Nix language.
Together with the usage of \texttt{nixpkgs} --- an extensive library of build and package declarations,
\texttt{mmc-image.nix} allows us to reliably and reproducibly build a bootable image of the complete software environment the project requires.
To then boot the generated image, it only needs to be flashed on a MultiMediaCard (MMC\footnote{Commonly referred to as: SD card, memory card.}) and slotted into the MMC-slot on the Raspberry Pi.

% Explain the pros of Nix
When building derivations (nomenclature for anything built with Nix: an executable binary, shared library file, a system environment, etc.) their dependencies are in complete isolation with each other, which effectively allows the avoidance of dependency hell.\footnote{Colloquial term referring to the frustration often generated when dealing with version-specific dependencies.
See \href{https://en.wikipedia.org/wiki/Dependency_hell}{Wikipedia}.}

% Explain the rollback functionality git provide us.
In combination with git, one may trivially roll back to previous derivations that are known to work by checking out a commit and rebuilding.

% Explain why treating the MMC as volatile is a good idea (MMCs have a tendency to just stop working).
In addition, by preferring a work flow where the target storage is considered volatile, any deficiencies of the target medium are mitigated.

% TODO: improve
\subsubsection{System-external services}
The software environment generated for the Raspberry Pi automatically connects to Eduroam if credentials are available.
Eduroam places some limitations on connected clients: firewall, e.g.
To enable easy remote access to the system, a reverse SSH proxy is established with a known bastion host which has a static IP address.
By exposing this proxy via a known port on the bastion, any system connected to the Internet may trivially access the Raspberry Pi remotely via a static endpoint.
While not a necessity for the project itself, this external service is a great convenience for ad-hoc experiments and general system debugging.

% Explain the content of contrib/bastion.nix
