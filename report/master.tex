\documentclass[twocolumn]{article}
\usepackage[english]{babel}
\usepackage[T1]{fontenc}
\usepackage{graphicx}
\usepackage{kpfonts}[maths]
\usepackage{libertine}
\usepackage{placeins}
\usepackage{gensymb}
\usepackage{inconsolata}
\usepackage{subcaption}
\usepackage{mdframed}
\usepackage{caption}
\usepackage{amsmath}
\usepackage{lipsum}
\usepackage{tikz}
\usepackage{minted}
\usepackage{lipsum}
\usepackage{xcolor}
\usepackage{sectsty}
\usepackage{enumitem}
\usepackage{hyperref}
\usepackage{csquotes}
\usepackage{biblatex}

\hypersetup{
    colorlinks=true,
    urlcolor=blue,
    linkcolor=black,
    citecolor=black,
}

\definecolor{CERNblue}{HTML}{22529e}
\definecolor{CodeBg}{rgb}{0.95, 0.95, 0.95}

\setminted{
    % linenos=true,
    autogobble=true,
    frame=single,
    % framesep=2mm,
    % framerule=0.4pt,
    tabsize=4,
    fontsize=\footnotesize,
    breaklines,
    bgcolor=CodeBg,
}

\setlist[enumerate]{itemsep=0mm}
\setlist[itemize]{itemsep=-0mm}


\sectionfont{\color{CERNblue}}
\subsectionfont{\color{CERNblue}}

\newcommand{\hypermail}[1]{\href{mailto:#1}{\texttt{<#1>}}}

% TODO: change footnote style

% TODO: add a link to the repository somewhere

\title{D/E7039E --- Project in Industrial Computer Systems/Electronical Systems \& Control Technology}
\author{
As authored by:
\footnote{In no particular order.}
\footnote{All team members can be contacted via
\href{mailto:vikson-6@student.ltu.se;lukkar-4@student.ltu.se;sansim-6@student.ltu.se;olemis-6@student.ltu.se;rubasp-6@student.ltu.se;alinou-6@student.ltu.se;alikhar-6@student.ltu.se}{this hyperlink}.} \\
Viktor Sonesten \hypermail{vikson-6@student.ltu.se} \\
Lukas Karlsson \hypermail{lukkar-4@student.ltu.se} \\
Simon Sandberg \hypermail{sansim-6@student.ltu.se} \\
Oleksiy Mishchenko \hypermail{olemis-6@student.ltu.se} \\
Ruben Asplund \hypermail{rubasp-6@student.ltu.se} \\
Ali Nouri \hypermail{alinou-6@student.ltu.se} \\
Ali Khademi \hypermail{alikha-6@student.ltu.se}
}
\date{\today}

\begin{document}
\maketitle

\appendix
\section{Team Members}
The team's members are below listed along with their preliminary areas of concern throughout the development of this project.

\begin{description}
    \item[Viktor Sonesten]
    Any embedded programming;
    repository maintainership; and
    report typesetting.

    \item[Lukas Karlsson]
    Embedded programming and
    Computer vision/position control.
    \item[Simon Sandberg]
    Position and grip control.
    Sensor data integration and tuning.
    Embedded programming.
    \item[Oleksiy Mishchenko]
    Sensor implementation and tuning;
    Implementation and tuning of control systems;
    Embedded programming;
    \item[Ruben Asplund]
    Arrowhead and embedded programming.
    \item[Ali Nouri]
    Embedded programming;
    Invers kinematics modeling;
    Choose/design a navigation system;
    \item[Ali Khademi]
    Dynamic equation
    Controller design for position
\end{description}

\section{Preliminary Way of Work}
All members will meet each workday in the lab at 9 AM if able to work on the project.

Every Monday after lunch a meeting will be held to wrap up the previous week and make any necessary preparations for the Tuesday presentation.

After each day it is recommended to write down what sub-projects one has worked on and what is planned for the next day/week;
this will greatly streamline the Monday meetings and ensure that every team member always have something to work on.

If we later find a better way to manage our work,
or if we find that we deviate from what is planned above,
an updated Way of Work will be appended to the appendix of this document.

\subsection{How the system is composed}
The current plan is to build the robot mostly with the lego ev3 parts. The mobile platform will be built with the parts present in the package given to the group in the beginning of the project. The arm, which will grip the cube, will be built from external lego parts and motors, the list of these parts will be submitted furing this Tuesday. A resberry pie will be used for the communication between sensors and motors, all calculations for the control systems will also be computed int it. A preliminary model has been calculated for the forward kinematics of the system, this will probably be updated and an inverse kinematic model will be derived. The decision has not yet been made on how the location of the robot will be calculatedl, some of the possible navigation systems that have been discussed are; LIDAR, using deckawave together with a sonar, using wifi triangulation and other solutions. 

\subsection{structure of work}
\subsubsection{Milestones}
\subsubsection{Tecchnical solutions in the project}




\end{document}
